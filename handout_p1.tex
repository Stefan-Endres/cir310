\documentclass[12pt, A4paper]{article}
\usepackage{hyperref}
\usepackage{booktabs}
\usepackage{graphicx}
\usepackage{amsmath}
\usepackage[utf8]{inputenc} 
\usepackage{pdflscape}

\usepackage{mhchem}
\usepackage[numbers]{natbib}

\usepackage{fancybox}
\usepackage{graphicx}
\usepackage{color}
%\usepackage[margin=0.7in]{geometry}
\setlength{\abovetopsep}{1em}

\usepackage{siunitx}
\sisetup{locale=ZA}


\usepackage[margin=1.2in]{geometry}

\graphicspath{{graph/}}

\title{CIR310 Phase equilibrium project\\Part 1}
\date{Submission deadline: 22.03.2017 before 13:30}

\newcounter{eqs}
\setcounter{eqs}{0}
\newcommand{\eq}{\stepcounter{eqs}\arabic{eqs}}

\newcounter{variables}
\setcounter{variables}{0}

\newcounter{inputs}
\setcounter{inputs}{0}
\newcommand{\definput}[1]{\ensuremath{#1}\stepcounter{inputs}\stepcounter{variables}}

\newcounter{outputs}
\setcounter{outputs}{0}
\newcommand{\defoutput}[1]{\ensuremath{#1}\stepcounter{outputs}\stepcounter{variables}}

\newcounter{parameters}
\setcounter{parameters}{0}
\newcommand{\defparameter}[1]{\ensuremath{#1}\stepcounter{parameters}\stepcounter{variables}}

\newcommand{\describesection}[1]{\multicolumn{2}{l}{\emph{#1}}}




\begin{document}
\maketitle
\section{System description}
A coupled reactor/column system is used for trioxane synthesis from formaldehyde to trioxane.  The reversible liquid phase chemical reaction is given by the trimerization reaction:
 
{\centering
\ce{ 3 CH_2O <=>[$[$H^+$]$] C_3H_3O_3 } \par
}
The equilibrium concentration of trioxane is very low in this reaction. A common design solution to increase overall conversion of the process is by using azeotropic distillation with water. A simplified model of this process was developed by \citet{HU19991353}. \autoref{fig:col} demonstrates this process.

 \begin{figure} \centering
 \includegraphics[scale=0.45]{img/reaccol.png}
 \caption{ The coupled reactor/column system (adapted from \citet{HU19991353}).} \label{fig:col}
\end{figure}
 
The feed stream of the reactor contains an aqueous solution of formaldehyde and water.
The goal of the process is to maximise the overall conversion of formaldehyde to trioxane.

\section{Model}
% % Assumptions
The following assumptions have been made for the process.
\begin{enumerate}
\item The entire reactor/column system operates at a continuous steady state equilibrium.
\item The reactor is controlled at a set temperature and the heat transfer dynamics is negligible.
\item The system is adiabatic so that no heat is lost to the environment.
\item The catalytic reaction occurs in the liquid phase of the reactor only. No reaction occurs in the column equilibrium stages.
\item Side reactions and intermediate products are negligible so that the system is a tertiary formaldyhyde-trioxane-water system.
\item A constant relatively volatility is assumed at all stages of the reaction.
\end{enumerate}

The system equations as developed by \citet{HU19991353} are shown in \autoref{tab:equations2}. %A description of parameter, input and output variable names as well as initial steady-state values are presented in Tables~\ref{tab:parameters},~\ref{tab:inputs} and~\ref{tab:outputs}.

%\include{solution_steps}

\begin{landscape}
\begin{table}[htbp]
  \centering
  \caption{System equations. Here $i$ is the chemical component index where $i = 1$ is formaldehyde, $i = 2$  is trioxane and $i=3$ is water. $n$ represents an equilibrium stage $n = 0, 1, 2, \dots N$, the stage $n = N$ is the reactor stage.}
  \label{tab:equations2}
  \begin{tabular}{rllll}
    \toprule
                                  & Equation                                            
                                  & Parameters          
                                  & Inputs         
                                  & Outputs  \\
    \midrule
    \describesection{Reactor mass balance} \\
    \eq                           & $F z_{i} + L x_{i, N} - V y_{i, N} + \upsilon_i r = 0$                           
                                  &   \defparameter{\upsilon_i}
                                  & \definput{F}  \definput{z_{i}}
                                  & \defoutput{ x_{i, N}  }   \defoutput{  y_{i, N}   }  \defoutput{ L }  \defoutput{ V }  \defoutput{  r}    \\
    \describesection{Column mass balance} \\
    \eq                           & $V y_{i, n} - (L + w_i D) x_{i, n} = 0$    
                                  &   \defparameter{w_i}
                                  & \definput{D}
                                  & \defoutput{x_{i, n}}  \defoutput{y_{i, n}} \\
    \eq                           & $L = R D$    
                                  &                       
                                  & \definput{R}
                                  & \\
    \describesection{Component continuity} \\
    \eq                           & $\sum^3_i z_{i} = 1$    
                                  &                       
                                  & 
                                  & \\
    \eq                           & $\sum^3_i x_{i,n} = 1$    
                                  &                       
                                  & 
                                  & \\
    \eq                           & $\sum^3_i y_{i,n} = 1$    
                                  &                       
                                  & 
                                  & \\
    \describesection{Phase equilibrium} \\
    \eq                           & $ \alpha_i = \frac{ y_{i,n} / y_{1, n} }{ x_{i, n+1} / x_{_{1, n + 1}} } $    
                                  &                       
                                  & 
                                  & \defoutput{\alpha_i} \\
    \describesection{Reaction rate equation} \\
    \eq                           & $ r = V_r \left( k_1 C_1^2 - k_2 C_2 \right)  $    
                                  &  \defparameter{k_1} \defparameter{k_2}       
                                  & 
                                  & \defoutput{r}\\
    \describesection{Energy balance} \\
    \eq                           & $ \sum^3_i \left[ Fz_i h_{F, i} + L x_{i, N} h_{l, i} - V y_{i, N} h_{v, i} \right] + Q = 0$    
                                 &                       
                                  & \definput{ h_{F, i} } \definput{ h_{l, i} } \definput{ h_{v, i} } \definput{ Q }
                                 & \\
    \midrule
                                  & Total: \arabic{variables} = & \arabic{parameters} & +\arabic{inputs} & +\arabic{outputs}  \\
    \bottomrule
  \end{tabular}
\end{table}

\end{landscape}



 
\newpage
\pagebreak
\section{Instructions}
Your company commissioned a reactor/column process based on a design using the described model.  After startup however, a much lower yield than predicted by the design was was found. You and your colleagues have been assigned to investigate the process model, to improve its accuracy and to find solutions to improve the yield of the process.  

\begin{enumerate}
\item Using  only your background in thermodynamics and elementary chemistry, list the possible reasons why a lower steady state conversion than predicted by the model was obtained. For every possibility listed, suggest solutions and/or tests that can be performed on the system. Where relevant show the improvements to the model (write down these equations). Provide at least 4 possibilities relevant to the phase equilibria of the system, especially with regards to the assumptions made by the model.  \ovalbox{40}
\item Prove that if an ideal gas model can be assumed, then increasing the reactor temperature will result in zero conversion of the reactant. Write down this temperature point. \ovalbox{10}
\item Describe how you would incorporate the following thermodynamic models into the system of equations. Write down the relevant supporting equations, but do not do any explicit symbolic computations. You may cite relevant algorithms from textbooks and other literature sources. Show that the degrees of freedom are zero. %\ovalbox{40}
\begin{itemize}
\item Activity coefficient models (ex. NRTL, UNIQUAC).  \ovalbox{20}
\item Cubic equation of state models (ex. Virial Van der Waals, Peng-Robinson, Redlich-Kwong). \ovalbox{20}
\item Virial equations of state (ex. Truncated compressibility equations with virial coefficient correlations). \ovalbox{5}
\item K-value correlations from VLE data. \ovalbox{5}
\end{itemize}

%\item \ovalbox{\verb|steady_state.py|} 
\end{enumerate}

\begin{centering}
Total Marks: \Ovalbox{100}\\
\end{centering}

%\pagebreak
\bibliographystyle{apalike}
%\bibliographystyle{spmpsci}
\bibliography{lib}

\end{document}
